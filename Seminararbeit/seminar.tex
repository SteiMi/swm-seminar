
\documentclass[runningheads,a4paper]{uwsese}

%ACHTUNG: Diesen Import für englische Arbeiten entfernen.
\usepackage[ngerman]{babel}

\usepackage[utf8]{inputenc}
\usepackage{amssymb}
\usepackage{amsmath}
\setcounter{tocdepth}{3}
\usepackage{graphicx}
\usepackage{array}
\usepackage{ragged2e}
\usepackage{arydshln}
\newcolumntype{P}[1]{>{\RaggedRight\hspace{0pt}}p{#1}}
%\newcolumntype{P}[1]{>{\raggedright\arraybackslash}p{#1}}


\usepackage{url}
\urldef{\mailsa}\path|maximilian.stock@stud-mail.uni-wuerzburg.de|
\newcommand{\keywords}[1]{\par\addvspace\baselineskip
\noindent\keywordname\enspace\ignorespaces#1}

\begin{document}

\mainmatter

% first the title is needed
\title{Konfigurationsmanagementstrategien und Tools in der SW Entwicklung:\\ Überblick und Beispiel}

% a short form should be given in case it is too long for the running head
\titlerunning{Konfigurationsmanagementstrategien und Tools}

\author{
  Maximilian Stock\\
  \texttt{maximilian.stock@stud-mail.uni-wuerzburg.de}\\
  \and\\
  Michael Steininger\\
  \texttt{michael.steininger@stud-mail.uni-wuerzburg.de}
}
%
%\authorrunning{Maximilian Stock}
% (feature abused for this document to repeat the title also on left hand pages)

% the affiliations are given next
\institute{Julius-Maximilians-Universität,
Würzburg\\
%\mailsa\\
}


% \toctitle{Interpolation based Modelling of Power and Performance}
% \tocauthor{J. von Kistowski and S. Kounevs}
\maketitle


\begin{abstract}
	Hier kommt der Abstract hin. Weglassen?
\end{abstract}


\section{Einführung}
Ursprünglich ist die Idee des Konfigurationsmanagements im US-Militär
entstanden. Dort wurden in den 1950er Jahren Versuche mit Flugkörpern gemacht,
die in verschiedenen Konfigurationen gebaut und getestet wurden. Eine
Vielzahl der Flugkörper explodierten nicht wie vorgesehen im Ziel, sondern
meist zuvor. Es war allerdings nicht möglich die fehlerhaften Flugkörper nach
den Tests zu untersuchen, weil sie zerstört wurden. Zusätzlich waren die
Aufzeichnungen über die Konfigurationen der einzelnen Flugkörper nicht
ausreichend. Dadurch war es nicht möglich festzustellen welche Änderungen zu
welchen Testausgängen führten. Aus diesen Erfahrungen enstanden
Konfigurationsmanagementstandards, die bei allen Änderungen eines Erzeugnisses
auch eine Änderung der Dokumente fordern.

Diese Probleme lassen sich analog auch in der Softwareentwicklung beobachten.
Es gibt z. B. Probleme durch undokumentierte Änderungen von Code oder
kurzfristige Änderungen am Quellcode vor Auslieferung. Deshalb ist das
Konfigurationsmanagement insbesondere in der
Automobilbranche Teil von Reifegradmodellen wie z. B. {\em CMMI} oder
{\em SPICE}. Damit überprüfen Automobilhersteller, ob ihre Zulieferer über ein
funktionierendes Konfigurationsmanagement verfügen~\cite[S. 2f]{weischedel2002}.

Zunächst wird im ersten Kapitel dieser Arbeit auf die Grundlagen des
Konfigurationsmanagement eingegangen. D. h. das Konfigurationsmanagement wird
definiert und erklärt.
Das zweite Kapitel erläutert wie man das allgemeine Konfigurationsmanagement in
der Softwareentwicklung anwendet.
In Kapitel drei wird das Konfigurationsmanagement an einem beispielhaften
Softwareprojekt in der Praxis gezeigt.

TODO: AUFBAU DER ARBEIT VOR ABGABE NOCHMAL ÜBERPRÜFEN OB ES NOCH PASST!!!!

\section{Grundlagen des Konfigurationsmanagement}
Im Folgenden wird Konfigurationsmanagement nach DIN EN ISO 10007 definiert und
dessen Tätigkeiten beschrieben. Es existieren eine Reihe von weiteren
Definitionen für Konfigurationsmanagement.
In dieser Arbeit wird im Folgenden die ISO Definition verwendet. Eine weitere
Definition ist z. B. im Buch ``Configuration Management Principles and
Practice'' von Hass beschrieben~\cite{Hass:2003:CMP:582584}. In den nächsten
Unterkapiteln werden die einzelnen Tätigkeiten des Konfigurationsmanagements
nach DIN EN ISO 10007 näher erläutert. Daraufhin folgt ein Unterkapitel über
Konfigurationsmanagementpläne.

Nach der Norm DIN EN ISO 10007 ist das Konfigurationsmanagement (kurz {\em KM})
wie folgt definiert:

``KM ist eine Managementdisziplin, die über die gesamte Lebensdauer eines
Erzeugnisses angewandt wird, um Transparenz und Überwachung seiner funktionellen
und physischen Merkmale sicherzustellen. Hauptziel von KM ist, die
gegenwärtige Konfiguration eines Erzeugnisses sowie den Stand der Erfüllung
seiner physischen und funktionellen Forderungen zu dokumentieren und volle
Transparenz herzustellen. Ein weiteres Ziel ist, dass jeder am Projekt
Mitwirkende zu jeder Zeit des Erzeugnislebenslaufs die richtige und zutreffende
Dokumentation verwendet. Der KM Prozess umfasst die folgenden integrierten
Tätigkeiten:

\begin{itemize}
	\item Konfigurationsidentifizierung
	\item Konfigurationsüberwachung
	\item Konfigurationsbuchführung
	\item Konfigurationsauditierung''~\cite{ISO10007}.
\end{itemize}

\subsection{Konfigurationsidentifizierung}
Bei der Konfigurationsidentifizierung geht es darum sog.
{\em Konfigurationseinheiten} festzulegen und zu beschreiben. Eine
Konfigurationseinheit kann dabei eine beliebige Kombination von Hardware,
Software und Dienstleistung sein. Für diese Einheiten werden in
Konfigurationsdokumenten die physischen und funktionellen Merkmale festgehalten.
Weiterhin wird bei der Konfigurationsidentifizierung festgelegt, nach welchen
Regeln bestimmte Artefakte nummeriert werden. Hier können Artefakte z. B.
Konfigurationseinheiten, ihre Teile und Zusammenstellungen von Dokumenten,
Schnittstellen und Änderungen sein. Außerdem werden sog.
{\em Bezugskonfigurationen} definiert. Diese sind mitsamt ihren Änderungen als
aktuell gültige Konfigurationen zu verstehen~\cite[S. 6f]{weischedel2002}.

\subsection{Konfigurationsüberwachung}
Die Konfigurationsüberwachung behandelt die Überwachung von Änderungen. Dabei
wird gefordert, dass alle Änderungen dokumentiert und begründet werden.
Zusätzlich sollten die Auswirkungen der Änderungen beurteilt werden. Auf Basis
dieser Beurteilungen sollen Änderungen genehmigt oder abgelehnt
werden~\cite[S. 7]{weischedel2002}.

\subsection{Konfigurationsbuchführung}
Die Tätigkeit Konfigurationsbuchführung sorgt für eine Rückverfolgbarkeit von
Änderungen. Damit soll es möglich sein, alle Änderungen zur letzten
Bezugskonfiguration nachvollziehen zu können. Aus der
Konfigurationsidentifizierung und der Konfigurationsüberwachung resultieren
die für die Buchführung notwendigen Aufzeichnungen als
Nebenprodukte~\cite[S. 7]{weischedel2002}.

\subsection{Konfigurationsauditierung}
Die Konfigurationsauditierung lässt sich in zwei Tätigkeiten aufteilen.

Durch das sog. {\em funktionsbezogene Konfigurationsaudit} wird gewährleistet,
dass das Erzeugnis den vertraglich spezifizierten Anforderungen entspricht.
D. h. die Konfigurationseinheit muss alle Leistungen und funktionellen Merkmalen
erreichen, die zuvor in den Konfigurationsdokumenten festgelegt wurden.

Das sog. {\em physische Konfigurationsaudit} prüft, ob eine aktuelle
Konfiguration einer Konfigurationseinheit mit den Konfigurationsdokumenten
übereinstimmt. Damit soll sichergestellt werden, dass die Dokumente und die
Erzeugnisse nicht voneinander abweichen. Ansonsten könnte passieren, dass
ein Erzeugnis letztendlich nichts mehr mit seiner Dokumentation zu tun
hat~\cite[S. 7]{weischedel2002}.

\subsection{Konfigurationsmanagementplan}
Der Konfigurationsmanagementprozess sollte in einem
{\em Konfigurationsmanagementplan} dokumentiert sein. Darin sollte für jedes
Projekt stehen welche Konfigurationsmanagementverfahren durchzuführen sind. Zu
jedem Verfahren soll im Plan festgelegt sein wer sie durchführt und wann sie
ausgeführt werden sollen~\cite[S. 7f]{weischedel2002}.

TODO: "Ein Beispiel für einen Konfigurationsmanagementplan ist in X zu sehen"

\section{Konfigurationsmanagement für Softwareentwicklung}

% sources
% https://en.wikipedia.org/wiki/Comparison_of_open-source_configuration_management_software
% https://en.wikipedia.org/wiki/Software_configuration_management

Konfigurationsmanagement lässt sich ebenfalls in der Softwareentwicklung
unter der Bezeichnung {\em software configuration management} (kurz SCM)
finden. Allgemein versteht man darunter Aktivitäten die zur Aufzeichnung und
Überprüfung von Änderungen in Softwareprojekten dienen. Dazu zählen unter
anderem Versionskontrolle und das Definieren von sogenannten {\em baselines}.
Baselines stellen hierbei klar definierte Zwischenstände eines Produkts dar, die
anhand der Attribute des Produkts unterschieden werden können. Ein
Entwicklungsschritt zwischen Baselines wird hierbei als {\em change} bezeichnet.
SCM ermöglicht es alle changes zu dokumentieren um festzuhalten was, wann
und durch wen geändert wurde. Im Falle einer fehlerhaften baseline kann per
SCM eine automatische Rückkonfiguration auf eine passendere, korrektere baseline
durchgeführt werden.

% cite  Roger S. Pressman (2009). Software Engineering: A Practitioner's Approach (7th International ed.). New York: McGraw-Hill.

\subsection{Build Management}
\subsection{Process Management}
\subsection{Environment Management}
\subsection{Teamwork???}
\subsection{Defect Tracking}

\section{Beispiel}
- Anwendung von expliziten Softwaretools auf ein Projekt

\section{Fazit}
- alles cool

%\newpage
\bibliographystyle{splncs03}
\bibliography{sesebib}

\end{document}
